\documentclass{article}
\usepackage{amsmath}
\usepackage[utf8]{inputenc}

\title{SI 710 Assignment 4: Experiment Design}
\author{Linfeng, Merve, Rosina, Tangren}
\date{}

\usepackage{natbib}
\usepackage{graphicx}
\usepackage{enumerate}
\usepackage[colorlinks,linkcolor=blue,bookmarksdepth=paragraph]{hyperref}

\begin{document}
\maketitle
% Old notes
%
\section{Opening notes on collaboration for this project}% To be removed when finished.

We can use this for collaborative document-editing.

\textbf{Note that}, there is also a chat feature, and the icon is at the right-margin of the window.

\section{Summary of the tasks}

In the lecture on Monday, there were two market mechanisms proposed. We are
asked to propose a workable experiment, in which the network effects may be
tested in form of market experiment.

Please add your insights on this. I am offering a skeleton only.

\section{Introducing Social Network}

Networks could be either induced by the experimenter, or brought into the
experiment from the real world.

\subsection{A quick random thought: Games shared through Facebook}

Facebook friends shall form a natural, if not real, social network. We
may devise an online-``game'' that is sharable through links onto Facebook.

Due to the asynchronistic nature of these online platform, it is hard to control
the size of games. There are two possibilities:
\begin{enumerate}
    \item We may believe in a wide spread of the game, if it is well designed.
        Then, a large sample shall get collected and we can sort out those games
        played by a certain number of ``friends'';
    \item We may well devise the market mechanism in a way that the solution
        concept is ``scale free'' from the number of participants. (Not sure if
        the market-design should still hold valid in this case.)
\end{enumerate}

Despite the drawbacks, as long as the initial nodes in the ``Facebook-network''
is drawn in a random-enough way, this snowball approach of sampling may provide
a network with several close clusters that is robust against biases.
As long as the subjects may coordinate into sizable groups simultaneously, we
are in an ideal experimental setting where subjects are visible to other
participants and they are recognizable through names by their friends.

\subsection{An induced social network approach}

\subsubsection{Self-selected network?}
In a lab-setting, we may have more power to ``manipulate'' subjects and have
them form various groups in a self-selection manner. For one possibility,
experimenter may propose a few conflicting categories and have subjects
self-select into those categories. The conflicting nature of these self-selected
groups, will then help form preferences over trading partners.

Nevertheless, this self-selected method may be no different from a mere testing
of social identity, which is lack of ``network structure''.

\subsubsection{Network through market design}

In the market design, we may implement several trading clusters, where sellers
and buyers may only see the price in this cluster. However, let there be several
brokers who may have access to prices (and thereby trading power) in other
clusters. Bestow some initial wealth for the brokers, and let the brokers have
the freedom of buying and selling, as long as their budget constraint is not
binding yet.

In this manner, artificial links are created and thereby a network is induced.
Though it remains a problem if this is a social network, we may argue that this
is the way the market really works: through interpersonal connections, trade may
happen. That is to say, unless one \textit{knows} the people there, no one in
another cluster will trade with him/her.

\subsubsection{Another possible thought on network structure}

So far, we have been taken the network structure as fixed. What if we add
network dynamics into the experiment? Not sure if this might worth discussing.

\section{Bullet points}

\subsection{Experiment Infrastructure}

\subsubsection{Computer based experiment}

\begin{itemize}
    \item Random ID for the control group: shuffling every period to eliminate
    any possible repeated game effect;
    \item Random role for members in the control group: also mitigating
    repeated-game effect.
\end{itemize}


\subsection{Market implementation}

\begin{itemize}
    \item Trading partners shall be able to ``seal the deal'', $\implies$ either we
    shall do it face-to-face, or a pretty delicate lab-experiment where tradings are
    broken into various stages.
    \begin{itemize}
        \item For implementation, we may adopt a sealed-auction scheme, where only the
        pair of buyer and seller is aware of the bids from both sides. (Like the hand-gestures
        for Japanese fish market.)
    \end{itemize}
    \item Sellers post prices publicly, buyers seek a seller and chat privately;
    \item Enable a single-buyer\&single-seller chat feature;
    \item Deals are made privately, and average is announced
\end{itemize}

\subsection{Network treatment}

\begin{itemize}
    \item Merve's shuffling method:
        \begin{itemize}
            \item For treatment group, induce networks via dividing subjects into groups
            of seller and buyers;
            \item For control, also do the group-division, but re-shuffle the groups
            (including the roles) after every trading period.
        \end{itemize}
\end{itemize}


\section{Remaining questions}


\section{Experiment Design}
\subsection{Motivation}
The purpose of this study is to see how the presence of social networks affects
trading patterns and, by extension, the efficiency of decentralized markets.


\subsubsection{Social Network Implementation}
Real world, existing friendships.

\subsection{Experiment Procedure}
\begin{enumerate}
    \item Randomly assign subjects into two groups, treatment and control.
        (Please refer to section ``Subjects, Control and Treatment'')
    \item Elicit network of treatment group (Please refer to section ``Network
        Elicitation Method'')
    \item Computer based trading: treatment with real ID; control with changing
        fake ID. Subject can post unbinding offers publicly but they can
        negotiate (private chatting) as well. (Please refer to section ``Market
        mechanism'')
    \item At the end of each round, subject will observe the average price or
        distribution(we can discuss which one is better) (just like
        ``truecar.com'')
    \item Subject will be reward the surplus they earned in the experiment.
\end{enumerate}

\subsubsection{Subjects, Control and Treatment}
The experiment will be run in a computer-based trading setting among a class of
undergraduate students, such as ECON 102. The main advantage of the selected
group of subjects is the easy access and the fact that we would not need
parental consent. This course is chosen since it is still a mass course and
provides students with enough time to establish a network in college. Hence, the
measured social network will be the existing friendship network among students.

In order to measure the importance of social networks on decentralized
competitive markets and the choice of trading partners, subjects are randomly
selected from the course and assigned into two groups of 25 individuals. The
size of each group is selected to make the elicitation mechanism possible. (if
not it is too long)

In the treatment group, students are identified with their real names which
implies that the subjects are able to recognized trading partners with whom they
have a closer social relationship. After the trading experiment is run, the
social network of the treatment group is going to be elicited. In the control
group, subjects are not identified and are randomly assign with a fake ID. Note
that in both groups, the market design is identical except from the fact that in
one group subjects are identifiable whereas in the other they are not.

The choice of this methodology allows us to identify how the social network
influences the trading patterns among subjects. We expect that subjects are more
likely to trade with individuals who belong to their social network.

\subsection{Market Mechanism}
We are focusing on a computer based double auction environment. The ask and bid
offers are going to be posted in a publicly available fashion. In contrast to
the transparency of the prices, agents will be able to privately chat and
negotiate. The transactions will be realized once both parties agree on it. We
will let sellers to have multiple units, whereas buyers are looking for one unit
of good to purchase. The parties will be charged a small amount that will be
determined with respect to the round of negotiations they make through private
chatting.

\subsection{Network Elicitation Method}

Given the contextual setting of lab-experiment, we are restricted to a network
of at most 25 subjects (max capacity of the lab in North Quad), which makes a
thorough elicitation of the social network possible. The goal is to elicit the
social network of subjects (both in control and treatment groups) as complete and
detailed as possible, with an emphasis of measuring the strength of ties, i.e.
acquaintances (weak ties), close friends (strong ties), \textit{etc.}

Inspired by Alan's experiment design, we here have experimenter nominate each
subject one at a time, and have the rest of subjects fill up a survey composed
of the following list of questions:
\begin{enumerate}
    \item How would you evaluate your relationship with him/her:
        \begin{enumerate}
            \item barely know each other;
            \item acquaintances;
            \item close friend;
        \end{enumerate}
        (Note here, Linfeng has omitted the category of ``friends''.)
    \item What is his/her major?

        A drop-down manual shall follow this question;
    \item Where does him/her come from?

        Open question box, and later we can check the matches at various levels
        (State, region, county and score the matches.)

    \item Are you in the same study group with him/her?

    \item Are you two in the same class other than Econ 102?

    \item What is him/her favorite (pick any one, or more, from the following)
        \begin{enumerate}
            \item movie;
            \item TV Show
            \item book
        \end{enumerate}

    \item TBD (may need to refer to Alan's questions for inspiration).
\end{enumerate}

This list of questions shall compose a length of one-full-screen that suits the
computers in the lab.

\subsubsection{Incentive structure}

Minimum incentive shall be provided, so as not to interfere with the market
game. Details to be settled when the full ``market game'' is settled, from which
we may calculate the expected payoff. As of now, I can only expect the total
payment from answering the survey to be less than, say, two dollars.

\subsubsection{Experimenter effect}

It should be well stated that, although the true names were used in the survey
and the market experiment (for treatment group), the experimenters will only
access the data through numeric IDs. This needs to be stated in a trust-worthy
way, so that the subjects do buy the argument and shall answer truthfully to the
``friendship survey''.


\subsection{Hypothesis and Testing}
\subsubsection{Hypotheses}
\begin{enumerate}[{Hypothesis} 1]
    \item Subjects will be more likely to trade with individuals who are in
        their social network.
    \item Bargaining costs will be lower when access to social networks is
        available.
    \item Efficiency will be higher in homogeneous good markets when networks
        are available. This is expected to be due to lower trading costs.
    \item Price dispersion will be greater in markets where social networks are
        available. In other words, the Law of One Price is less likely to hold.
    \item Surplus will be split more equally in markets with social networks.
    \item Friends with stronger ``ties'' will trade more items than friends with
        weak ``ties'', if anything at all.
\end{enumerate}

\subsubsection{Testing Hypothesis}
\begin{enumerate}[{Hypothesis} 1]
    \item  Utilizing the elicited network, compute the frequency to within network
        transaction of treatment group, compare it with no network effect
        probability.
    \item  Compute means and standard deviations of bargaining time of control
        group, treatment group ( a.total transactions; b.transactions within
        network; c. transactions outside of network), then calculate $p$-value to check
        statistic significance.
    \item  Compute total surplus achieved in each round in both control and
        treatment, then calculate the statistic difference of these two.
    \item Compute means and standard deviations of price dispersion in each round of
        control group, treatment group ( a.total transactions; b.transactions within
        network; c. transactions outside of network), then calculate $p$-value to check
        statistic significance.
    \item Compute means and standard deviations of surplus splitting ratio (sell
            surplus/buyer surplus) in each round of control group, treatment group (
            a.total transactions; b.transactions within network; c. transactions outside of
        network), then calculate $p$-value to check statistic significance.
\end{enumerate}








\bibliographystyle{plain}
\bibliography{references}
\end{document}
