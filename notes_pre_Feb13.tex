
\section{Opening notes on collaboration for this project}% To be removed when finished.

We can use this for collaborative document-editing.

\textbf{Note that}, there is also a chat feature, and the icon is at the right-margin of the window.

\section{Summary of the tasks}

In the lecture on Monday, there were two market mechanisms proposed. We are
asked to propose a workable experiment, in which the network effects may be
tested in form of market experiment.

Please add your insights on this. I am offering a skeleton only.

\section{Introducing Social Network}

Networks could be either induced by the experimenter, or brought into the
experiment from the real world.

\subsection{A quick random thought: Games shared through Facebook}

Facebook friends shall form a natural, if not real, social network. We
may devise an online-``game'' that is sharable through links onto Facebook.

Due to the asynchronistic nature of these online platform, it is hard to control
the size of games. There are two possibilities:
\begin{enumerate}
    \item We may believe in a wide spread of the game, if it is well designed.
        Then, a large sample shall get collected and we can sort out those games
        played by a certain number of ``friends'';
    \item We may well devise the market mechanism in a way that the solution
        concept is ``scale free'' from the number of participants. (Not sure if
        the market-design should still hold valid in this case.)
\end{enumerate}

Despite the drawbacks, as long as the initial nodes in the ``Facebook-network''
is drawn in a random-enough way, this snowball approach of sampling may provide
a network with several close clusters that is robust against biases.
As long as the subjects may coordinate into sizable groups simultaneously, we
are in an ideal experimental setting where subjects are visible to other
participants and they are recognizable through names by their friends.

\subsection{An induced social network approach}

\subsubsection{Self-selected network?}
In a lab-setting, we may have more power to ``manipulate'' subjects and have
them form various groups in a self-selection manner. For one possibility,
experimenter may propose a few conflicting categories and have subjects
self-select into those categories. The conflicting nature of these self-selected
groups, will then help form preferences over trading partners.

Nevertheless, this self-selected method may be no different from a mere testing
of social identity, which is lack of ``network structure''.

\subsubsection{Network through market design}

In the market design, we may implement several trading clusters, where sellers
and buyers may only see the price in this cluster. However, let there be several
brokers who may have access to prices (and thereby trading power) in other
clusters. Bestow some initial wealth for the brokers, and let the brokers have
the freedom of buying and selling, as long as their budget constraint is not
binding yet.

In this manner, artificial links are created and thereby a network is induced.
Though it remains a problem if this is a social network, we may argue that this
is the way the market really works: through interpersonal connections, trade may
happen. That is to say, unless one \textit{knows} the people there, no one in
another cluster will trade with him/her.

\subsubsection{Another possible thought on network structure}

So far, we have been taken the network structure as fixed. What if we add
network dynamics into the experiment? Not sure if this might worth discussing.

\section{Bullet points}

\subsection{Experiment Infrastructure}

\subsubsection{Computer based experiment}

\begin{itemize}
    \item Random ID for the control group: shuffling every period to eliminate
    any possible repeated game effect;
    \item Random role for members in the control group: also mitigating
    repeated-game effect.
\end{itemize}


\subsection{Market implementation}

\begin{itemize}
    \item Trading partners shall be able to ``seal the deal'', $\implies$ either we
    shall do it face-to-face, or a pretty delicate lab-experiment where tradings are
    broken into various stages.
    \begin{itemize}
        \item For implementation, we may adopt a sealed-auction scheme, where only the
        pair of buyer and seller is aware of the bids from both sides. (Like the hand-gestures
        for Japanese fish market.)
    \end{itemize}
    \item Sellers post prices publicly, buyers seek a seller and chat privately;
    \item Enable a single-buyer\&single-seller chat feature;
    \item Deals are made privately, and average is announced
\end{itemize}

\subsection{Network treatment}

\begin{itemize}
    \item Merve's shuffling method:
        \begin{itemize}
            \item For treatment group, induce networks via dividing subjects into groups
            of seller and buyers;
            \item For control, also do the group-division, but re-shuffle the groups
            (including the roles) after every trading period.
        \end{itemize}
\end{itemize}


\section{Remaining questions}
