
Given the contextual setting of lab-experiment, we are restricted to a network
of at most 25 subjects (max capacity of the lab in North Quad), which makes a
thorough elicitation of the social network possible. The goal is to elicit the
social network of subjects (both in control and treatment groups) as complete and
detailed as possible, with an emphasis of measuring the strength of ties, i.e.
acquaintances (weak ties), close friends (strong ties), \textit{etc.}

Inspired by Alan's experiment design, we here have experimenter nominate each
subject one at a time, and have the rest of subjects fill up a survey composed
of the following list of questions:
\begin{enumerate}
    \item How would you evaluate your relationship with him/her:
        \begin{enumerate}
            \item barely know each other;
            \item acquaintances;
            \item close friend;
        \end{enumerate}
        (Note here, Linfeng has omitted the category of ``friends''.)
    \item What is his/her major?

        A drop-down manual shall follow this question;
    \item Where does him/her come from?

        Open question box, and later we can check the matches at various levels
        (State, region, county and score the matches.)

    \item Are you in the same study group with him/her?

    \item Are you two in the same class other than Econ 102?

    \item What is him/her favorite (pick any one, or more, from the following)
        \begin{enumerate}
            \item movie;
            \item TV Show
            \item book
        \end{enumerate}

    \item TBD (may need to refer to Alan's questions for inspiration).
\end{enumerate}

This list of questions shall compose a length of one-full-screen that suits the
computers in the lab.

\subsubsection{Incentive structure}

Minimum incentive shall be provided, so as not to interfere with the market
game. Details to be settled when the full ``market game'' is settled, from which
we may calculate the expected payoff. As of now, I can only expect the total
payment from answering the survey to be less than, say, two dollars.

\subsubsection{Experimenter effect}

It should be well stated that, although the true names were used in the survey
and the market experiment (for treatment group), the experimenters will only
access the data through numeric IDs. This needs to be stated in a trust-worthy
way, so that the subjects do buy the argument and shall answer truthfully to the
``friendship survey''.