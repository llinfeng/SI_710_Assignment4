
\section{Notes from Lecture 7 on Feb 15, feedbacks + thoughts }

\subsection{Feedback on the experiment design proposed as draft}
Written in a tone of a group project, with a group meeting on Feb 21, 2016 in
head.

\subsubsection{Problems}

\begin{enumerate}
    \item How to let the subjects in the treatment group recognize their
        friends?
        \begin{itemize}
            \item If through photos, then other traits shall penetrate: say,
                girls would like to trade with girls, and boys may offer
                discount for beautiful ladies \& \textit{vise versa}.
            \item $\implies$ maybe, we may use the real names to
                for agents to recognize each other?

                May consider introduce the confounding ``human-recognition''
                step-by-step:
                \begin{enumerate}
                    \item For Treatment Group I: use only the name;
                    \item For Treatment Group II: use name + photos (or photos
                        only)
                \end{enumerate}
                To further elicit the effect of photos, may:
                \begin{itemize}
                    \item[(c)] Introduce another control group, where agents with
                        random IDs are identified through (randomly assigned?)
                        photos.
                \end{itemize}
        \end{itemize}
    \item What would the chatting channel brings:
        \begin{itemize}
            \item Chatting feature may introduce various confounds\footnote{\textcolor{green}{Chatting is already costly via:\\
            \begin{enumerate}
                \item they might miss a good price in the market while chatting
                \item they might be embarrassed to ask for a lower/higher price
            \end{enumerate}} }. Two ways out:
                \begin{enumerate}
                    \item Implement the chat feature in full, and adopt NLP
                        (natural language processing)?
                    \item Remove the chat feature, and let agents only interact
                        through numeric bids \textcolor{green}{Maybe we can have a hypothesis about chatting}.
                \end{enumerate}
        \end{itemize}
    \item Alan's list of questions: missing
        \begin{itemize}
            \item Emails sent to Alan echoed no reply. We may need to prolong
                the network-eliciting survey together.
        \end{itemize}
    \item Could other measure of the elicited network matter for trading
        patterns?
        \begin{itemize}
            \item Had the agents been able to trade more than one good, will the
                distribution of deals be more evenly spread when the set of
                agents involved in the trade belong to a more \textbf{dense}
                network?
            \item Given this, does the clustering, and diameter of network
                matter, when we are observing a vivid market with multiple
                friends trading?
        \end{itemize}
        \item \textcolor{green}{How to make 24 participants appear?
        \begin{itemize}
            \item The amount that will be paid to the participants for an hour, should in expectation be equal to the hourly wage of tutoring (depends on the participant pool, what the alternative occupation wage would be?)
            \item Also, people usually put more effort into the experiment when they are negotiating over larger amount of money, so maybe let's have high numbers/valuations/prices and let's do the payments in laboratory units.
        \end{itemize}}

        \item \textcolor{green}{Maybe put a restriction of, say 10 rounds of negotiating through chatting, instead of the cost that is imposed for the private chatting rounds.}

       \item \textcolor{green}{We expect a convergence-how many rounds should we have in each session?}

       \item Subjects: 401 undergraduates instead of 101,102? Since there are stronger, denser network within 401 class.

\end{enumerate}

